\documentclass[11pt]{report}
\usepackage[utf8]{inputenc}


\usepackage{amsmath}



\begin{document}

\chapter*{Heat equation}

The heat equation is in carthesian coordinates is derived from Fourier's law of heat conduction and 
is written as
\begin{equation}
\frac{\partial}{\partial\mathbf{x}}\left(k\frac{\partial T}{\partial \mathbf{x}} \right) + \dot{e}_{gen} = \rho c\frac{\partial T}{\partial t}
\end{equation}

where $T$ is the variable temperature, $\dot{e}_{gen}$ is the constant rate of heat generated per unit volume,
 $k$ is the material thermal conductivity, $\rho$ is the matrial density and $c$ is the specific heat.
For a constant material conductivity a parameter $\alpha = \frac{k}{\rho c}$ is usualy introduced and it describes
the thermal diffusivity.


\section*{Boundary conditions}

For the heat equation there are many possible boundary conditions
that may be applied. Which ones and how many needed are problem dependent.





\section*{Deriving exact solutions}

For simple geometries and assumptions, exact solutions can be derived.



\section*{A finite difference approach}

A second order scheme for the one dimensional heat equation with constant
can be written as

\begin{equation}
\frac{T_{i+1}^n-2T_i^n+T_{i-1}^n}{\Delta t^2} = \frac{1}{\alpha}\frac{T_i^{n+1}-T_i^{n-1}}{2\Delta t}
\end{equation}





\end{document}
